\section{Escalabilitat}

L'arquitectura del CPD ha estat dissenyada de tal forma que es pugui escalar segons les necessitats del servei. Un exemple, si creix el volum d'usuaris i el sistema ha de processar més peticions web es podría incrementar la quantitat de Frontals en conjunt amb la connexió a internet. Un altre exemple, en cas de requerir més espai d'emmagatzematge es podría incrementar amb nodes de NFS-JBOD

Per no produir colls d'ampolla s'hauria d'incrementar les conexions de xarxa que es vegin afectades introduint switchos necessaris i contractant els requeriments d'internet si es veuen afectats.

Un altre punt a tenir en compte: si el sistema incrementa el seu volum de negoci substancialment seria convenient estudiar la possibilitat de canviar de collocation center per ajustar-lo als nous requisits i intentar estalviar costos.

El sistema de Backup-Mirror allotjat a un altre CPD podria oferir més tipus de node per incrementar si fos necessari la disponibilitat del servei com fan altres organitzacions.

Una altra possibilitat de cara al futur seria substituir switchos que ofereixen conexió Ethernet mitjançant RJ45, per tecnologies més ràpides, probablement els costos d'aquests dispositius es veuran reduits i la substitució esdevindrà recomanada.